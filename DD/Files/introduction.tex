% ===================================================================
% SECTION 1: INTRODUCTION
% ===================================================================

\subsection{Purpose}
The purpose of this Design Document (DD) is to provide a comprehensive technical specification for the \textbf{Best Bike Paths (BBP)} system. While the Requirement Analysis and Specification Document (RASD) focused on the functional and non-functional requirements from a user's perspective, this document describes the software architecture, the decomposition into components, and the design choices made to satisfy those requirements.

The DD serves as a primary reference for the development team, providing the blueprints necessary for the implementation phase. It bridges the gap between the problem description (RASD) and the final source code.

\subsection{Scope}
The BBP system is a modular platform designed to support individual cyclists. The scope of this design document covers the following technical areas:
\begin{itemize}
    \item \textbf{Mobile Client Application}: A cross-platform application responsible for user interaction, real-time GPS tracking, and path visualization.
    \item \textbf{Backend Infrastructure}: A RESTful web service handling authentication, data persistence, and the business logic for calculating path safety scores.
    \item \textbf{External Integrations}: The interfaces with third-party Map Providers (for rendering tiles and routing) and Meteorological Services (for atmospheric data enrichment).
\end{itemize}

\subsection{Definitions, Acronyms, Abbreviations}
\begin{itemize}
    \item \textbf{API}: Application Programming Interface.
    \item \textbf{REST}: Representational State Transfer, an architectural style for distributed hypermedia systems.
    \item \textbf{DTO}: Data Transfer Object, an object that carries data between processes.
    \item \textbf{DAO}: Data Access Object, an object that provides an abstract interface to some type of database.
    \item \textbf{JWT}: JSON Web Token, a compact, URL-safe means of representing claims to be transferred between two parties.
    \item \textbf{DBMS}: Database Management System.
    \item \textbf{HTTPS}: Hypertext Transfer Protocol Secure.
\end{itemize}

\subsection{Revision History}
\begin{table}[h!]
    \centering
    \begin{tabu} to \textwidth { X[0.4,l,p] X[0.6,l,p] X[2,l,p] }
        \hline
        \textbf{Version} & \textbf{Date} & \textbf{Description} \\
        \hline
        1.0 & \today & Initial version of the Design Document. \\
        \hline
    \end{tabu}
    \caption{Revision history of the DD}
\end{table}

\subsection{Reference Documents}
\begin{itemize}
    \item Politecnico di Milano, \emph{Assignment RDD AY 2024-2025 - Best Bike Paths}.
    \item Haipeng Zhu, \emph{Best Bike Paths - Requirement Analysis and Specification Document (RASD)}, 2024.
    \item IEEE Std 1016-2009, \emph{IEEE Standard for Information Technology - Systems Design - Software Design Descriptions}.
\end{itemize}

\subsection{Document Structure}
The document is organized into the following main sections:
\begin{itemize}
    \item \textbf{Section 2 - Architectural Design}: Provides a high-level overview of the system architecture, component views, deployment strategies, and runtime behavior.
    \item \textbf{Section 3 - User Interface Design}: Describes the user experience and provides mockups for the mobile application.
    \item \textbf{Section 4 - Requirements Traceability}: Maps each functional requirement from the RASD to specific design elements in the DD.
    \item \textbf{Section 5 - Implementation, Integration \& Testing}: Outlines the development plan and the strategy for verifying system correctness.
\end{itemize}