
\subsection{Purpose}
The purpose of this Design Document (DD) is to provide a comprehensive technical specification for the \textbf{Best Bike Paths (BBP)} system. While the Requirement Analysis and Specification Document (RASD) focused on the functional and non-functional requirements from a user's perspective, this document describes the software architecture, the decomposition into components, and the design choices made to satisfy those requirements.

The DD serves as a primary reference for the development team, providing the blueprints necessary for the implementation phase. It bridges the gap between the problem description (RASD) and the final source code.

\subsection{Scope}
The BBP system is a modular platform designed to support individual cyclists. The scope of this design document covers the following technical areas:
\begin{itemize}
    \item \textbf{Mobile Client Application}: A cross-platform application responsible for user interaction, real-time GPS tracking, and path visualization.
    \item \textbf{Backend Infrastructure}: A RESTful web service handling authentication, data persistence, and the business logic for calculating path safety scores.
    \item \textbf{External Integrations}: The interfaces with third-party Map Providers (for rendering tiles and routing) and Meteorological Services (for atmospheric data enrichment).
\end{itemize}

\subsection{Definitions, Acronyms, Abbreviations}
\begin{itemize}
    \item \textbf{API}: Application Programming Interface.
    \item \textbf{REST}: Representational State Transfer, an architectural style for distributed hypermedia systems.
    \item \textbf{DTO}: Data Transfer Object, an object that carries data between processes.
    \item \textbf{DAO}: Data Access Object, an object that provides an abstract interface to some type of database.
    \item \textbf{JWT}: JSON Web Token, a compact, URL-safe means of representing claims to be transferred between two parties.
    \item \textbf{DBMS}: Database Management System.
    \item \textbf{HTTPS}: Hypertext Transfer Protocol Secure.
\end{itemize}

\subsection{Revision History}
\begin{table}[H]
    \centering
    \begin{tabular}{|l|l|p{8cm}|}
        \hline
        \textbf{Version} & \textbf{Date}   & \textbf{Description}                                                                                                                                                             \\ \hline
        1.0              & January 8, 2026 & Initial submission of the Design Document.                                                                                                                                       \\ \hline
        2.0              & January 9, 2026 & \textbf{Major Revision (Post-Review)}: Comprehensive upgrade of architectural diagrams, UI fidelity, and algorithm specifications. Detailed changes are listed in Section 1.4.1. \\ \hline
    \end{tabular}
    \caption{Revision history of the DD}
    \label{tab:revision_history}
\end{table}

\subsubsection{Changelog: Enhancements in Version 2.0}
This version represents a significant iteration over the initial submission, focusing on technical depth and architectural clarity. The key improvements include:

\begin{itemize}
    \item \textbf{Architectural Refinement}:
          \begin{itemize}
              \item Upgraded \textbf{Component} and \textbf{Deployment Diagrams} to include specific protocols (HTTPS, JDBC), port numbers, and execution environments (Docker), replacing generic blocks with industry-standard notations.
              \item Introduced explicit interfaces (APIs) and defined the "Offline-First" synchronization logic.
          \end{itemize}

    \item \textbf{Advanced Runtime Logic}:
          \begin{itemize}
              \item Rewrote all \textbf{Sequence Diagrams (UC1-UC6)} to include error handling (\texttt{alt} blocks), asynchronous processing (\texttt{par/group}), and security validations (\texttt{ref} to UC5), moving beyond simple "happy paths".
              \item Added a detailed \textbf{State Diagram} covering the complex lifecycle of the "Safety Score" visibility.
          \end{itemize}

    \item \textbf{High-Fidelity User Interface}:
          \begin{itemize}
              \item Replaced wireframes with \textbf{High-Fidelity Mockups} featuring real Milan geospatial data (Polimi $\rightarrow$ Duomo route).
              \item Implemented "Situational Design" principles with high-contrast modes and specific safety alerts (e.g., Tram Tracks).
          \end{itemize}

    \item \textbf{Technical Specifications}:
          \begin{itemize}
              \item Added the \textbf{Safety Score Algorithm} pseudocode (Section 2.9) to mathematically define how user reports are aggregated.
              \item Included specific JSON Data Contracts for the core RESTful endpoints.
          \end{itemize}
\end{itemize}

\subsection{Reference Documents}
\begin{itemize}
    \item Politecnico di Milano, \emph{Assignment RDD AY 2024-2025 - Best Bike Paths}.
    \item Haipeng Zhu, \emph{Best Bike Paths - Requirement Analysis and Specification Document (RASD)}, 2024.
    \item IEEE Std 1016-2009, \emph{IEEE Standard for Information Technology - Systems Design - Software Design Descriptions}.
\end{itemize}

\subsection{Document Structure}
The document is organized into the following main sections:
\begin{itemize}
    \item \textbf{Section 2 - Architectural Design}: Provides a high-level overview of the system architecture, component views, deployment strategies, and runtime behavior.
    \item \textbf{Section 3 - User Interface Design}: Describes the user experience and provides mockups for the mobile application.
    \item \textbf{Section 4 - Requirements Traceability}: Maps each functional requirement from the RASD to specific design elements in the DD.
    \item \textbf{Section 5 - Implementation, Integration \& Testing}: Outlines the development plan and the strategy for verifying system correctness.
\end{itemize}

\section*{Appendix: Use of AI Tools}
In accordance with the academic integrity guidelines, the following AI tools were utilized to assist in the drafting and refinement of this Design Document.

\subsection*{1. Text Refinement \& Content Structuring}
\begin{itemize}
    \item \textbf{Tool}: Large Language Models (ChatGPT-4 / Gemini)
    \item \textbf{Usage}: Used to refine the technical English phrasing of the "Architectural Design" and "Design Rationale" sections to ensure professional tone and clarity.
    \item \textbf{Prompt Example}: \textit{"Refine the following description of the Offline-First strategy to highlight data consistency and synchronization logic."}
\end{itemize}

\subsection*{2. Diagram Generation Code}
\begin{itemize}
    \item \textbf{Tool}: LLM for PlantUML Generation
    \item \textbf{Usage}: Generated the initial PlantUML code structures for Sequence Diagrams (UC1-UC6) and the Component Diagram based on the logical flow descriptions.
    \item \textbf{Prompt Example}: \textit{"Generate a PlantUML sequence diagram for a User Login flow implementing JWT stateless authentication with error handling."}
\end{itemize}

\subsection*{3. User Interface Prototyping}
\begin{itemize}
    \item \textbf{Tool}: AI-assisted Code Generation (Tailwind CSS / HTML)
    \item \textbf{Usage}: Generated the HTML/CSS code to visualize the high-fidelity UI mockups for the Dashboard and Recording screens, which were then rendered and included as figures.
    \item \textbf{Rationale}: This allowed for rapid prototyping of the "Situational Design" paradigm without manual pixel-pushing in vector tools.
\end{itemize}