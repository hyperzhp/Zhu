% ===================================================================
% SECTION 5: IMPLEMENTATION, INTEGRATION & TESTING
% ===================================================================

\subsection{Overview}
This section outlines the strategy for the development and verification of the \textbf{Best Bike Paths (BBP)} system. The implementation follows an incremental approach, while the testing strategy adopts the "Testing Pyramid" model to ensure robustness at the unit, integration, and system levels.



\subsection{Implementation Plan}
The development process is organized into four major milestones (Sprints), focusing on delivering a minimum viable product (MVP) before expanding to complex analytical features.

\subsubsection{Feature Identification and Development Phases}
\begin{itemize}
    \item \textbf{Sprint 1: Foundation (Backend \& Auth)}
        \begin{itemize}
            \item Setup PostgreSQL database with PostGIS extensions.
            \item Implement User Controller and JWT-based authentication.
            \item Develop the API Gateway and base routing logic.
        \end{itemize}
    \item \textbf{Sprint 2: Mobile Recording Infrastructure}
        \begin{itemize}
            \item Implement the high-frequency GPS sampling engine on the Mobile Client.
            \item Develop local SQLite persistence for the "Offline-First" strategy.
            \item Create basic Map dashboard using external tile providers.
        \end{itemize}
    \item \textbf{Sprint 3: Integration \& External Services}
        \begin{itemize}
            \item Implement the asynchronous Trip Service to handle trajectory uploads.
            \item Integrate with the External Weather API for automated enrichment.
            \item Enable the Path Reporter module for manual status submissions.
        \end{itemize}
    \item \textbf{Sprint 4: Analytics \& UI Polishing}
        \begin{itemize}
            \item Develop the Path Safety Scorer algorithm on the backend.
            \item Implement the Map Visualizer overlays for scored paths.
            \item Final UX/UI refinement and end-to-end performance optimization.
        \end{itemize}
\end{itemize}

\subsection{Component Integration and Testing}
Integration testing focuses on the interaction between the Mobile Client and the Cloud Backend, as well as the connectivity with external map and weather providers.

\begin{itemize}
    \item \textbf{API Integration}: Automated tests (e.g., using Postman/Newman) to verify that the Backend correctly parses complex GeoJSON trajectories sent by the client.
    \item \textbf{Service Resiliency}: Testing the system's behavior when external APIs (Weather/Map) are slow or unreachable, ensuring the core trip-saving logic remains functional.
    \item \textbf{Data Consistency}: Verifying that private/public visibility toggles correctly filter data in the database during cross-module queries.
\end{itemize}

\subsection{System Testing}
System testing evaluates the BBP application in a real-world environment to ensure it meets all functional and non-functional requirements defined in the RASD.

\subsubsection{Functional Correctness}
Field tests will be conducted in urban environments (e.g., central Milan) to verify that recorded distances match reference odometers and that obstacles are reported at the correct geographic coordinates.

\subsubsection{Non-Functional Performance Testing}
\begin{itemize}
    \item \textbf{GPS Accuracy \& Battery Consumption}: Monitoring the impact of continuous GPS polling (1Hz) on device battery life over a 2-hour cycling session.
    \item \textbf{Concurrency}: Simulating multiple users uploading trips simultaneously to test the backend's response time and database locking behavior.
    \item \textbf{Recovery}: Simulating app crashes during a recording session to verify that the Local Cache Manager correctly recovers the partial trip data upon restart.
\end{itemize}

\subsection{Additional Specifications on Testing}
Due to the safety-critical nature of "Best Bike Paths," particular emphasis is placed on \textbf{Geospatial Accuracy Testing}. This involves validating that the Map Engine correctly aligns the recorded GPS points with the underlying road network topology (Map Matching) to prevent misleading safety score visualizations.