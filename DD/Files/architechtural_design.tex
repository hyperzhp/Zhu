\section{Architectural Design}

\subsection{Architectural Design Overview}
The Best Bike Paths (BBP) system is designed as a client--server application.
A mobile client provides user interaction and lightweight local management, while a backend server exposes REST APIs for persistence and shared services integration.
A dedicated database server stores users, trips, bike paths, and related information.
External services (Google Maps API and OpenWeatherMap API) are integrated to support map visualization and contextual information.

The chosen architecture aims to:
\begin{itemize}
  \item isolate UI concerns from business logic and persistence concerns;
  \item enable independent evolution of mobile client and backend services;
  \item keep the system scalable enough for future extensions while remaining simple for the single-/small-group scope.
\end{itemize}

\subsection{Component View}
Figure~\ref{fig:component-diagram} shows the main software components of BBP and their dependencies.

\begin{figure}[H]
  \centering
  \includegraphics[width=\textwidth]{Images/component_diagram.png}
  \caption{Component Diagram -- Best Bike Paths (BBP).}
  \label{fig:component-diagram}
\end{figure}

\subsubsection{Mobile Client}
The Mobile Client (Android/iOS) is responsible for user interaction and for orchestrating user-facing flows.
It contains the following logical components:

\paragraph{UI Layer}
The UI Layer renders screens and collects user inputs.
It triggers application flows for trip recording, path reporting, and path visualization.

\paragraph{Trip Manager}
The Trip Manager coordinates the lifecycle of trip recording on the client side (start, stop, confirmation to save).
It sends trip data to the backend and optionally stores a local cache for quick access.

\paragraph{Path Reporter}
The Path Reporter supports manual insertion of path-related information (e.g., path condition, obstacles) and visibility preferences.
It validates user input and submits it to the backend.

\paragraph{Map Visualizer}
The Map Visualizer displays map tiles and overlays for possible bike paths between origin and destination.
It requests map tiles from Google Maps API and asks the backend for path-related metadata.

\paragraph{Local Data Manager}
The Local Data Manager caches recent trips and user preferences to reduce latency and support basic offline browsing (optional, depending on scope).
Cached data must never be treated as authoritative; the backend remains the source of truth.

\subsubsection{Backend Server}
The Backend Server exposes REST APIs and contains the business logic.
It consists of:

\paragraph{API Gateway}
Single entry point for the mobile client.
It performs request routing, basic validation, and authentication checks (based on the project scope).

\paragraph{Trip Controller}
Implements trip-related application services, such as storing a recorded trip and retrieving a user’s history.

\paragraph{Path Controller}
Manages operations related to bike paths and path information.
This includes creation of path information and enforcing visibility rules.

\paragraph{User Controller}
Handles user-related operations needed by the app, such as retrieving a profile and enforcing authorization constraints.

\paragraph{Weather Service Integration}
Encapsulates integration with OpenWeatherMap API.
It is used only when the application requires weather context for visualization or trip review.
If not required by scope, this component can be kept as a placeholder for future evolution.

\subsubsection{Data Layer}
The Data Layer persists the system state in a PostgreSQL database.
Core entities stored include:
\begin{itemize}
  \item Users and basic profile data;
  \item Trips (including timestamps, geometry/polyline, summary attributes);
  \item BikePaths and PathInformation (including visibility flags).
\end{itemize}

\subsubsection{External Services}
\paragraph{Google Maps API}
Used by the mobile client to display map tiles and route geometry overlays.

\paragraph{OpenWeatherMap API}
Used by the backend integration component to retrieve weather information for a given location and time, if required.

\subsection{Deployment View}
Figure~\ref{fig:deployment-diagram} shows the physical deployment of BBP.

\begin{figure}[H]
  \centering
  \includegraphics[width=0.65\textwidth]{Images/deployment_diagram.png}
  \caption{Deployment Diagram -- Best Bike Paths (BBP).}
  \label{fig:deployment-diagram}
\end{figure}

The mobile application runs on the user’s smartphone and communicates with the backend API via HTTPS over the Internet.
The backend is deployed on a cloud server (e.g., AWS) and interacts with a dedicated database server over TCP/IP.
External services are accessed over HTTPS by the mobile client (Google Maps) and by the backend integration component (OpenWeatherMap).

\subsection{Architectural Styles and Patterns}
BBP adopts:
\begin{itemize}
  \item a \textbf{client--server} style (mobile client + backend);
  \item a \textbf{layered} separation inside the backend (API gateway/controllers/data access);
  \item \textbf{integration adapter} for external services (Weather Service Integration).
\end{itemize}

\subsection{Design Rationale}
The architecture is intentionally simple to match the course scope:
\begin{itemize}
  \item it supports clear separation of responsibilities (UI vs business logic vs persistence);
  \item it enables independent testing of backend services and client flows;
  \item it makes future extensions feasible (e.g., multi-user aggregation, conflict resolution, optimization strategies) without redesigning the entire system.
\end{itemize}
