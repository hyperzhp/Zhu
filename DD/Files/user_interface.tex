% ===================================================================
% SECTION 3: USER INTERFACE DESIGN
% ===================================================================

\subsection{Design Rationale}
The User Interface (UI) of the \textbf{Best Bike Paths (BBP)} application is designed following the "Situational Design" paradigm. Recognizing that the primary users (cyclists) will interact with the application in diverse outdoor conditions—including direct sunlight, high-speed movement, and potential environmental distractions—the UI prioritizes \textbf{readability}, \textbf{efficiency}, and \textbf{safety}.

The following principles guide the interface design:
\begin{itemize}
    \item \textbf{High Contrast \& Legibility}: Utilizing a high-contrast color palette (primarily deep blues and stark whites) to ensure map elements and safety scores are visible under varying lighting conditions.
    \item \textbf{One-Handed Interaction}: Key actionable elements, such as the "Start/Stop Recording" and "Quick Report" buttons, are placed within the "Natural Thumb Zone" (bottom 30\% of the screen) to facilitate safe usage while stationary or during brief stops.
    \item \textbf{Minimal Cognitive Load}: Information density is kept low during active recording, displaying only mission-critical data (speed, distance, and immediate path alerts).
\end{itemize}

\subsection{Interaction Flow and Navigation}
The application utilizes a \textbf{Flat Navigation} structure to ensure users can reach core features with a maximum of two taps from the home screen.

\begin{enumerate}
    \item \textbf{Map-Centric Dashboard}: Upon successful authentication, the user is presented with a full-screen map interface. This acts as the central hub for exploration and activity initiation.
    \item \textbf{Recording Overlay}: When a trip is initiated, the dashboard transitions into an "Active Mode." The map persists in the background, but a semi-transparent overlay displays real-time telemetry.
    \item \textbf{Modal Reporting System}: To submit path information, a bottom-sheet modal is utilized. This allows the user to categorize road conditions (Optimal, Sufficient, Poor) via large, icon-based buttons, minimizing the need for text input.
    \item \textbf{Dynamic Feedback}: The UI provides haptic and visual confirmation for every successful data sync or report submission, acknowledging the "Offline-First" nature of the system.
\end{enumerate}

\subsection{Functional Screen Descriptions}

\subsubsection{Main Map Dashboard}
The dashboard integrates several critical UI components:
\begin{itemize}
    \item \textbf{Search Bar}: Located at the top for destination input.
    \item \textbf{Floating Action Button (FAB)}: A prominent "Start Trip" button situated at the bottom-center.
    \item \textbf{Safety Layers}: A toggleable layer that color-codes road segments (Green, Yellow, Red) based on the Safety Scoring Engine results.
\end{itemize}

\subsubsection{Trip Recording Interface}
During an active session, the interface undergoes a contextual transformation:
\begin{itemize}
    \item \textbf{Telemetry HUD}: Displays current speed (km/h) and elapsed distance in high-visibility sans-serif fonts.
    \item \textbf{Safety Alerts}: Persistent icons that appear if the user is approaching a segment with a low safety score or reported obstacles.
    \item \textbf{Finalization View}: After stopping a trip, the user is presented with a summary card showing the route taken, average speed, and a status indicator for the asynchronous weather data fetch.
\end{itemize}

\subsubsection{Path Reporting Interface}
The reporting flow is designed to be completed in less than 10 seconds:
\begin{itemize}
    \item \textbf{Category Grid}: A 3x2 grid of predefined obstacle types (e.g., Pothole, Heavy Traffic, Blocked Lane).
    \item \textbf{Visibility Toggle}: A simple switch to decide whether the information should be shared publicly to update the global safety score or kept as a private note.
\end{itemize}

\subsection{Accessibility and Safety Considerations}
To comply with modern accessibility standards, the BBP application supports:
\begin{itemize}
    \item \textbf{Dynamic Type}: Interface elements scale according to system-wide font size settings.
    \item \textbf{Voice Feedback}: Optional audio cues for path safety alerts, allowing cyclists to receive information without glancing at the screen.
    \item \textbf{Safe-Mode Restriction}: Certain non-essential features (e.g., detailed trip history browsing) are discouraged or simplified when the GPS sensor detects speeds above a certain threshold (e.g., 5 km/h).
\end{itemize}