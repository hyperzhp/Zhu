\subsection{External Interface Requirements}
\subsection{External Interface Requirements}

\subsubsection{User Interfaces}
The user interface of the BBP application shall be designed to be intuitive, responsive, and accessible, considering that users might interact with it while outdoors.
\begin{itemize}
    \item \textbf{Dashboard Map View}: The main screen shall display a map centered on the user's current location, showing nearby bike paths and their status codes (e.g., Green for optimal, Red for maintenance required).
    \item \textbf{Trip Recording Interface}: A simplified high-contrast screen displaying real-time metrics (Duration, Distance, Current Speed) with large "Pause" and "Stop" buttons to facilitate interaction during stops.
    \item \textbf{Reporting Form}: An input form for manually adding path information, featuring dropdown menus for "Status" and "Obstacles" to minimize typing effort.
    \item \textbf{Trip History}: A chronological list of past trips, allowing users to tap and view detailed statistics and weather data.
\end{itemize}
All UI components shall adhere to the Material Design (Android) and Human Interface Guidelines (iOS) standards.

\subsubsection{Hardware Interfaces}
The system requires direct interaction with the following hardware components of the mobile device:
\begin{itemize}
    \item \textbf{GPS/GNSS Sensor}: The system shall interface with the device's location services to obtain geolocation coordinates (latitude, longitude, altitude) with a minimum accuracy of 10 meters.
    \item \textbf{Storage}: The system shall access the device's internal storage to cache map data and temporarily save trip logs before synchronization.
    \item \textbf{Network Interface}: The system shall utilize the device's Wi-Fi or Cellular Data (4G/5G) interface to communicate with the backend server and external APIs.
\end{itemize}

\subsubsection{Software Interfaces}
\begin{itemize}
    \item \textbf{Map Service API}: The system shall interface with an external map provider (e.g., OpenStreetMap or Google Maps API) to render map tiles and perform geocoding.
    \item \textbf{Weather Service API}: The system shall communicate with a third-party meteorological service (e.g., OpenWeatherMap) via REST API to retrieve weather conditions based on timestamp and location.
    \item \textbf{Mobile Operating System}: The application shall be compatible with Android 12+ and iOS 15+, utilizing native OS APIs for background location tracking.
\end{itemize}

\subsubsection{Communication Interfaces}
\begin{itemize}
    \item \textbf{Protocol}: All client-server communication shall be encrypted using TLS 1.3 (HTTPS) to ensure data confidentiality.
    \item \textbf{Data Format}: Data exchange regarding trips, paths, and user profiles shall be formatted in JSON (JavaScript Object Notation).
\end{itemize}
\subsubsection{User Interfaces}
The BBP system shall provide a mobile application interface for cyclists. Key screens include:
\begin{itemize}
    \item \textbf{Dashboard/Map View}: The main screen displaying the user's current location and nearby bike paths on a map.
    \item \textbf{Trip Recording Interface}: Displays real-time statistics (speed, duration) and controls to start/stop recording.
    \item \textbf{Trip History}: A list of previously recorded trips with summary data.
    \item \textbf{Path Information Form}: A form to manually insert data about a path (status, obstacles) and set visibility.
\end{itemize}
The UI shall be designed to be intuitive and usable on touch-screen devices.

\subsubsection{Hardware Interfaces}
The system requires interaction with the following hardware components of the user's mobile device:
\begin{itemize}
    \item \textbf{GPS Receiver}: To track the user's geolocation (latitude, longitude) during trip recording.
    \item \textbf{Touch Screen}: For user input and interaction.
    \item \textbf{Network Interface}: (4G/5G/Wi-Fi) To communicate with the backend server and external services.
\end{itemize}

\subsubsection{Software Interfaces}
\begin{itemize}
    \item \textbf{Map Service API}: The system shall interface with an external map provider (e.g., Google Maps API or OpenStreetMap) to render maps and visualize paths.
    \item \textbf{Weather Service API}: The system shall interface with an external meteorological service to retrieve weather conditions (temperature, wind, rain) associated with a recorded trip.
    \item \textbf{Mobile Operating System}: The application shall be compatible with major mobile OS platforms (e.g., Android, iOS).
\end{itemize}

\subsubsection{Communication Interfaces}
\begin{itemize}
    \item The client application shall communicate with the BBP backend server using \textbf{HTTPS} to ensure data security.
    \item Data exchange format shall be \textbf{JSON} for structured information transfer.
\end{itemize}



\subsection{Functional Requirements}

The following is a list of functional requirements derived from the system goals.

\paragraph{Trip Recording}
\begin{itemize}
    \item \textbf{R1}: The system shall allow the user to start and stop the recording of a trip.
    \item \textbf{R2}: The system shall track the user's geolocation via GPS during the trip.
    \item \textbf{R3}: The system shall calculate trip statistics, including total distance and average speed.
    \item \textbf{R4}: The system shall retrieve weather data from an external service upon trip completion.
    \item \textbf{R5}: The system shall store the recorded trip details and statistics in the user's history.
\end{itemize}

\paragraph{History Management}
\begin{itemize}
    \item \textbf{R6}: The system shall allow users to view a list of their past trips.
    \item \textbf{R7}: The system shall allow users to view detailed information for a specific past trip.
\end{itemize}

\paragraph{Path Information Management}
\begin{itemize}
    \item \textbf{R8}: The system shall allow users to manually insert information about a bike path (street name, status, obstacles).
    \item \textbf{R9}: The system shall allow users to set the visibility of their inserted information to "Private" or "Publishable".
\end{itemize}

\paragraph{Path Visualization}
\begin{itemize}
    \item \textbf{R10}: The system shall allow users to input an origin and a destination address.
    \item \textbf{R11}: The system shall identify and visualize possible bike paths between the origin and destination on a map.
    \item \textbf{R12}: The system shall compute and display a score for each path based on its status and user reports.
\end{itemize}

\subsubsection{Use Cases}

% UC1: Record Trip
% Covers the requirement: "The recording of personal trips." 
% Includes weather data retrieval 
\begin{longtable}{|p{3.5cm}|p{11cm}|}
    \hline
    \textbf{Name}            & \textbf{UC1: Record a Biking Trip}                                                   \\ \hline
    \textbf{Actors}          & Registered User, External Weather Service                                            \\ \hline
    \textbf{Entry Condition} & The user is logged into the BBP application, and the device's GPS is active.         \\ \hline
    \textbf{Event Flow}      &
    1. The User navigates to the "Trip" section and presses "Start Recording". \newline
    2. The System begins tracking the user's location and duration. \newline
    3. The User rides their bike to the destination. \newline
    4. The User presses "Stop Recording". \newline
    5. The System calculates trip statistics (distance, average speed). \newline
    6. The System requests current weather data from the External Weather Service. \newline
    7. The System saves the trip with statistics and weather data to the user's history. \newline
    8. The System displays the trip summary to the User.                                                            \\ \hline
    \textbf{Exit Condition}  & The trip is successfully stored in the database and visible in the user's history.   \\ \hline
    \textbf{Exceptions}      &
    (6) \textbf{Weather Service Unavailable}: If the external service does not respond, the System saves the trip without weather data and notifies the user. \newline
    (2) \textbf{GPS Signal Lost}: If GPS is lost during the ride, the System pauses recording and prompts the user. \\ \hline
\end{longtable}

\vspace{0.5cm}

% UC2: Manual Info Insertion
% Covers the requirement: "The insertion of publishable information in manual mode." 
% Covers specific fields (street name, status) 
\begin{longtable}{|p{3.5cm}|p{11cm}|}
    \hline
    \textbf{Name}            & \textbf{UC2: Insert Path Information (Manual)}                                                                \\ \hline
    \textbf{Actors}          & Registered User                                                                                               \\ \hline
    \textbf{Entry Condition} & The user is logged into the BBP application.                                                                  \\ \hline
    \textbf{Event Flow}      &
    1. The User selects ``Add Path Info''. \newline
    2. The User manually enters the name of the street(s) included in the path. \newline
    3. The User selects the status of the path (e.g., optimal, sufficient, requires maintenance). \newline
    4. The User optionally describes obstacles (e.g., potholes). \newline
    5. The User toggles the visibility setting (Publishable or Private). \newline
    6. The User confirms the submission. \newline
    7. The System validates the input and stores the information.                                                                            \\ \hline
    \textbf{Exit Condition}  & The path information is stored. If marked ``Publishable'', it becomes available for system-wide path scoring. \\ \hline
    \textbf{Exceptions}      &
    (2) \textbf{Invalid Street Name}: The System cannot verify the street name against the map database and asks the User to correct it.     \\ \hline
\end{longtable}

\vspace{0.5cm}

% UC3: Visualize Paths
% Covers the requirement: "The visualization of the information about possible paths..." 
% Covers origin/destination and scoring 
\begin{longtable}{|p{3.5cm}|p{11cm}|}
    \hline
    \textbf{Name}            & \textbf{UC3: Visualize Bike Paths}                                                   \\ \hline
    \textbf{Actors}          & User (Registered or Unregistered)                                                    \\ \hline
    \textbf{Entry Condition} & The user has the application open on the map view.                                   \\ \hline
    \textbf{Event Flow}      &
    1. The User specifies an "Origin" and a "Destination". \newline
    2. The User requests to visualize paths. \newline
    3. The System computes available paths between the points. \newline
    4. The System calculates a score for each path based on path status and effectiveness. \newline
    5. The System displays the paths on the map, highlighting the one with the highest score. \newline
    6. The User browses the path details.                                                                           \\ \hline
    \textbf{Exit Condition}  & The best available bike paths are visualized on the map.                             \\ \hline
    \textbf{Exceptions}      &
    (3) \textbf{No Path Found}: The System cannot find a viable bike path between the points and notifies the User. \\ \hline
\end{longtable}

% UC4: User Registration
% Necessary because only "Registered users" can record trips 
\begin{longtable}{|p{3.5cm}|p{11cm}|}
    \hline
    \textbf{Name}            & \textbf{UC4: User Registration}                                        \\ \hline
    \textbf{Actors}          & Unregistered User                                                      \\ \hline
    \textbf{Entry Condition} & The user has installed the application but does not have an account.   \\ \hline
    \textbf{Event Flow}      &
    1. The User launches the application and selects "Sign Up". \newline
    2. The User enters required details (email, password, username). \newline
    3. The System validates the input format. \newline
    4. The System creates a new account in the database. \newline
    5. The System confirms successful registration and logs the user in.                              \\ \hline
    \textbf{Exit Condition}  & A new user profile is created, and the user is authenticated.          \\ \hline
    \textbf{Exceptions}      &
    (3) \textbf{Email Already Exists}: The System notifies the user that the email is already in use. \\ \hline
\end{longtable}

\vspace{0.5cm}

% UC5: User Login
% Standard functionality found in the example RASD 
\begin{longtable}{|p{3.5cm}|p{11cm}|}
    \hline
    \textbf{Name}            & \textbf{UC5: User Login}                                                                                \\ \hline
    \textbf{Actors}          & User                                                                                                    \\ \hline
    \textbf{Entry Condition} & The user opens the application and is not authenticated.                                                \\ \hline
    \textbf{Event Flow}      &
    1. The User enters email and password. \newline
    2. The User presses the "Login" button. \newline
    3. The System verifies the credentials against the database. \newline
    4. The System grants access and redirects the User to the main dashboard.                                                          \\ \hline
    \textbf{Exit Condition}  & The User is authenticated and can access restricted features (e.g., Record Trip).                       \\ \hline
    \textbf{Exceptions}      &
    (3) \textbf{Invalid Credentials}: The System displays an error message ("Incorrect email or password") and asks the user to retry. \\ \hline
\end{longtable}

\vspace{0.5cm}

% UC6: View Trip History
% Covers the requirement to "keep track of their cycling activities" 
\begin{longtable}{|p{3.5cm}|p{11cm}|}
    \hline
    \textbf{Name}            & \textbf{UC6: View Trip History}                                                                   \\ \hline
    \textbf{Actors}          & Registered User                                                                                   \\ \hline
    \textbf{Entry Condition} & The user is logged in and has previously recorded at least one trip.                              \\ \hline
    \textbf{Event Flow}      &
    1. The User navigates to the "My Trips" (History) section. \newline
    2. The System retrieves the list of past trips associated with the user account. \newline
    3. The System displays the list, showing summary data (date, distance) for each trip. \newline
    4. The User selects a specific trip to view details. \newline
    5. The System displays detailed statistics (average speed, duration, weather conditions) for the selected trip.              \\ \hline
    \textbf{Exit Condition}  & The User successfully reviews their past cycling activities.                                      \\ \hline
    \textbf{Exceptions}      &
    (2) \textbf{No History Available}: If the user has no recorded trips, the System displays a "No trips recorded yet" message. \\ \hline
\end{longtable}

\subsubsection{Sequence Diagrams}

The following diagram illustrates the interaction flow for recording a biking trip.

\begin{figure}[H]
    \centering
    \includegraphics[width=0.9\textwidth]{UC1.png}
    \caption{Sequence Diagram: UC1 - Record a Biking Trip}
    \label{fig:uc1_sequence}
\end{figure}

The process begins when the user initiates the recording...


The following diagram describes how a user manually inserts information regarding a bike path.

\begin{figure}[H]
    \centering
    \includegraphics[width=0.9\textwidth]{UC2.png}
    \caption{Sequence Diagram: UC2 - Insert Path Information}
    \label{fig:uc2_sequence}
\end{figure}


This diagram illustrates the process of requesting and visualizing the best bike path between two points.

\begin{figure}[H]
    \centering
    \includegraphics[width=0.9\textwidth]{UC3.png}
    \caption{Sequence Diagram: UC3 - Visualize Bike Paths}
    \label{fig:uc3_sequence}
\end{figure}


The registration process flow, including validation checks, is shown below.

\begin{figure}[H]
    \centering
    \includegraphics[width=0.9\textwidth]{UC4.png}
    \caption{Sequence Diagram: UC4 - User Registration}
    \label{fig:uc4_sequence}
\end{figure}

The authentication flow for user login is depicted in the following diagram.

\begin{figure}[H]
    \centering
    \includegraphics[width=0.9\textwidth]{UC5.png}
    \caption{Sequence Diagram: UC5 - User Login}
    \label{fig:uc5_sequence}
\end{figure}


The following diagram describes the interaction for retrieving and viewing past trip records, including statistics and weather data.

\begin{figure}[H]
    \centering
    \includegraphics[width=0.9\textwidth]{UC6.png}
    \caption{Sequence Diagram: UC6 - View Trip History}
    \label{fig:uc6_sequence}
\end{figure}

\subsubsection{Requirement Mapping}
The following table maps the Goals (G) identified in the Introduction to the Functional Requirements (R) and Domain Assumptions (D).

\begin{longtable}{|p{2cm}|p{8cm}|p{4cm}|}
    \hline
    \textbf{Goal}           & \textbf{Related Requirements} & \textbf{Assumptions} \\ \hline
    \textbf{G1} (Record)    & R1, R2, R3, R4                & D1, D2, D5           \\ \hline
    \textbf{G2} (History)   & R5                            & D3                   \\ \hline
    \textbf{G3} (Visualize) & R10, R11, R12                 & D3, D6               \\ \hline
    \textbf{G4} (Add Info)  & R6, R7, R8                    & D4                   \\ \hline
    \textbf{G5} (Share)     & R9                            & D4                   \\ \hline
\end{longtable}

\subsection{Performance Requirements}

The BBP system shall provide responses to user interactions within a reasonable time frame.

In particular:
\begin{itemize}
    \item Trip recording activation and termination shall be acknowledged within a few seconds.
    \item Visualization of bike paths shall be completed within an acceptable delay, depending on network conditions.
\end{itemize}

\subsection{Design Constraints}

\subsubsection{Standards Compliance}
\begin{itemize}
    \item \textbf{GDPR Compliance}: Since the system collects personal data (location history, email), it must comply with the General Data Protection Regulation (GDPR) to ensure user privacy.
    \item \textbf{Map Data Standards}: The system handles geospatial data and should adhere to standard formats (e.g., GeoJSON) for compatibility with map services.
\end{itemize}

\subsubsection{Hardware Limitations}
\begin{itemize}
    \item \textbf{GPS Accuracy}: The system's tracking precision is limited by the accuracy of the mobile device's GPS sensor, which may vary in urban canyons or bad weather.
    \item \textbf{Battery Consumption}: Continuous GPS tracking is battery-intensive. The application must be optimized to minimize battery drain during long trips.
\end{itemize}

\subsubsection{Other Constraints}
\begin{itemize}
    \item \textbf{Network Dependency}: Core features like map visualization and weather retrieval require an active internet connection. Offline capabilities are limited to local recording (if supported by design).
\end{itemize}

\subsection{Software System Attributes}

\subsubsection{Reliability}
The BBP system shall ensure that recorded trips and inserted path information are not lost once confirmed by the user.

\subsubsection{Availability}
The system shall be available during normal operation times, subject to network connectivity and external service availability.

\subsubsection{Security}
The system shall restrict access to personal data and ensure that only authenticated users can access their recorded trips and private path information.

\subsubsection{Maintainability}
The BBP system shall be designed to allow future extensions, such as additional path attributes or enhanced visualization features.

\subsubsection{Portability}
The system shall be deployable on commonly used platforms supporting web or mobile applications.


% =================================================
% =================================================
% 4. Formal Analysis Using Alloy
% =================================================
