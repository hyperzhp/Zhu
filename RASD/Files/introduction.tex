\subsection{Purpose}
The purpose of this document is to provide a Requirement Analysis and Specification for the Best Bike Paths (BBP) system.

BBP is a software system designed to support cyclists in recording their personal biking activities and in managing the route information about bike paths. The system allows users to manually insert information about bike paths, such as their status and the presence of obstacles, and to visualize possible bike paths between a origin starting point and a destination on a map.

This document defines the goals, assumptions, and requirements of the BBP system. It serves as a contractual reference between stakeholders and developers and as a baseline for subsequent design and implementation activities.

\subsubsection{Goals}
%Describe the main goals of the system
G1: The user wants to record a biking trip, in order to keep track of personal biking activities.

G2: The user wants to review previously recorded biking trips, in order to recall and organize past experiences.

G3: The user wants to visualize possible biking paths between a specified origin and destination, in order to support route selection.

G4: The user wants to manually add information about a bike path based on personal experience, in order to document relevant conditions and obstacles.

G5: The user wants to decide whether the information added about a bike path should be made visible to other users, in order to control its sharing.
\subsection{Scope}
The scope of this document encompasses the core software architecture and functional logic required to support individual cyclists in their daily activities.

\subsubsection{In-Scope Functionalities}
The system focuses on the following key areas:
\begin{itemize}
    \item \textbf{Telemetry Recording}: The real-time tracking of geolocation data to reconstruct biking trips, calculating metrics such as distance, duration, and average speed.
    \item \textbf{Crowdsourced Data Management}: The mechanism for users to manually report static path conditions (e.g., surface quality) and transient obstacles (e.g., roadworks).
    \item \textbf{Path Visualization \& Scoring}: The rendering of bike paths on a map interface, enriched with user-contributed data and computed scores reflecting path quality.
    \item \textbf{Privacy Controls}: Granular control mechanisms allowing users to toggle the visibility of their contributed data between private and public scopes.
\end{itemize}

\subsubsection{Out-of-Scope Functionalities}
To maintain a feasible scope for the current development phase (single-student project), the following features are explicitly excluded:
\begin{itemize}
    \item \textbf{Algorithmic Route Optimization}: The system will not perform Dijkstra or A* pathfinding algorithms to suggest new routes; it relies on pre-calculated routes from external APIs.
    \item \textbf{Social Networking}: Features such as "friending" other users, commenting on trips, or leaderboards are not included.
    \item \textbf{Hardware Integration}: Direct integration with external hardware sensors (e.g., heart rate monitors, cadence sensors) via Bluetooth/ANT+ is excluded.
\end{itemize}
\subsubsection{World Phenomena}
% Describe phenomena outside the system control
World phenomena are events and conditions that occur independently of the BBP system and are not directly controlled by it. These phenomena belong to the environment in which the system operates.

\begin{itemize}
    \item WP1: The user starting and stopping the recording of a biking trip through the system.
    \item WP2: The user manually inserting information about a bike path, including its status and possible obstacles.
    \item WP3: The system storing recorded biking trips.
    \item WP4: The system displaying recorded trips and bike paths on a map.
    \item WP5: The user specifying an origin and a destination to visualize possible biking paths.
    \item WP6: The user deciding whether the information added about a bike path is visible to other users.
\end{itemize}

\subsubsection{Shared Phenomena}
% Describe phenomena shared between users and the system
Shared phenomena are events and information that are observable and managed both by the users and the BBP system. These phenomena define the interaction between the system and its environment.

\begin{itemize}
    \item SP1: The user signals the system to start and stop the recording of a trip.
    \item SP2: The system retrieves meteorological data from an external weather service to enrich trip records.
    \item SP3: The user manually inputs street names, path status, and obstacle descriptions into the system.
    \item SP4: The user sets a specific trip or path report as "publishable" or "private" within the application.
    \item SP5: The system provides the user with trip statistics, including total distance and average speed.
    \item SP6: The user inputs an origin and a destination to request path options.
    \item SP7: The system visualizes bike paths on a map, displaying their computed scores to the user.
    \item SP8: The system provides a list/inventory of the user's previously recorded trips for review.
\end{itemize}

\subsection{Definitions, Acronyms, Abbreviations}
\begin{itemize}
    \item BBP: Best Bike Paths.
    \item User: An individual who interacts with the BBP system to record biking activities and manage bike path information.
    \item Trip: A biking performed by a user and recorded through the BBP system.
    \item Bike Path: A designated route for biking, which can be recorded and managed within the BBP system.
    \item Publishable Information: Information about bike paths that users can choose to share with others through the BBP system.
\end{itemize}
\subsection{Revision History}

\begin{longtable}{lll}
    \toprule
    Version & Date   & Description     \\
    \midrule
    1.0     & \today & Initial version \\
    \bottomrule
\end{longtable}

\subsection{Reference Documents}
\begin{itemize}
    \item Sotfware Engineering 2 - Requirement Engineering and Design Assignment Description
    \item IEEE/ISO/IEC 29148-2018 - Systems and software engineering -- Life cycle processes -- Requirements engineering
\end{itemize}
\subsection{Document Structure}
This document is organized as follows.

Section 2 provides an overall description of the BBP system, including scenarios, user characteristics, and domain assumptions.

Section 3 detailes the functional and non-functional requirements of the system.

Section 4 presents a formal analysis of selected aspects of the system using Alloy.

Section 5 reports the effort spent on this document.
% =================================================
