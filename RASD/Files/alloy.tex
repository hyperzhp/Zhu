In this section, we present a formal specification of critical aspects of the Best Bike Paths (BBP) system using \textbf{Alloy}, a declarative modeling language based on first-order logic and relational calculus.

\subsection{Methodological Approach}
Given the complexity of state management in mobile applications, particularly regarding data persistence and privacy, natural language requirements can sometimes be ambiguous. To mitigate this, we employed formal methods to:
\begin{enumerate}
    \item \textbf{Validate Data Consistency}: Ensuring that relationships between users, trips, and path reports obey cardinality and visibility constraints (Static Model).
    \item \textbf{Verify Temporal Logic}: Ensuring that the lifecycle of a trip follows a strict, irreversible sequence of states, preventing invalid system states such as "recording a trip that hasn't started" (Dynamic Model).
\end{enumerate}

\subsection{Static Structural Model}
The static model captures the immutable rules governing the data entities. A key design decision in BBP is the separation of "Private" and "Shared" data.

\begin{verbatim}
// ---------------------------------------------------------
// MODEL 1: STATIC STRUCTURE
// ---------------------------------------------------------

sig User {}
sig BikePath {}

sig Trip {
    user: one User,
    path: one BikePath
}

sig PathInformation {
    author: one User,
    path: one BikePath,
    visibility: one Visibility
}

enum Visibility {
    Private,
    Shared
}

run {} for 5
\end{verbatim}

\subsubsection{Visualization of Static Model}
We executed the static model to generate instances that verify the relationships between the entities. Figure \ref{fig:alloy_static} illustrates a generated world showing the connections between Users, Trips, and the Path Information they author.

\begin{figure}[H]
    \centering
    % This corresponds to your uploaded image: image_69ba1c.png
    % Please rename it to alloy_static.png
    \includegraphics[width=1.0\textwidth]{alloy_static.png}
    \caption{Alloy Instance: Static relationships between Users, Trips, and Path Information}
    \label{fig:alloy_static}
\end{figure}

\subsection{Dynamic Behavioral Model}
The dynamic model focuses on the \texttt{Trip} entity, which is the system's most state-sensitive component. We utilized the \texttt{util/ordering} module to simulate time steps.

The critical property we wished to verify is the \textbf{integrity of the recording process}. A trip must logically flow from an \texttt{Idle} state (system ready), through a \texttt{Recording} state (data accumulation), to a \texttt{Recorded} state (persistence).

\begin{verbatim}
// ---------------------------------------------------------
// MODEL 2: DYNAMIC BEHAVIOR (Trace)
// ---------------------------------------------------------

module bbp_trace

open util/ordering[Time]

enum TripState { Idle, Recording, Recorded }

sig Time {}

sig User {}
sig BikePath {}

sig Trip {
    user: one User,
    path: one BikePath,
    state: Time -> one TripState
}

fun st[t: Trip, tm: Time]: one TripState {
    t.state[tm]
}

// F1: Initial State - All trips start as Idle
fact InitialState {
    all t: Trip | st[t, first] = Idle
}

// F2: Lifecycle Constraints
// Defines valid transitions between time steps
fact TripLifecycle {
    all t: Trip, tm: Time - last | {
        let s  = st[t, tm] |
        let s2 = st[t, tm.next] |
        (s = Idle      implies (s2 = Idle or s2 = Recording)) and
        (s = Recording implies (s2 = Recording or s2 = Recorded)) and
        (s = Recorded  implies (s2 = Recorded))
    }
}

// ASSERTION: Consistency Check
// Verifies that a trip cannot be 'Recorded' without having been 'Recording'
assert RecordedImpliesWasRecording {
    all t: Trip |
        (some tm: Time | st[t, tm] = Recorded) implies
        (some tm: Time | st[t, tm] = Recording)
}

check RecordedImpliesWasRecording for 6 but exactly 4 Time

// PREDICATE: Simulation
// Generates a trace where a trip successfully reaches the Recorded state
pred EventuallyRecorded {
    some t: Trip | st[t, last] = Recorded
}

run EventuallyRecorded for 6 but exactly 4 Time
\end{verbatim}

\subsubsection{Analysis Results}
We executed the model using the Alloy Analyzer to validate the system logic.

\paragraph{Assertion Checking}
The assertion \texttt{RecordedImpliesWasRecording} was checked within a scope of 6 atoms and 4 time steps.
\begin{verbatim}
Executing "Check RecordedImpliesWasRecording for 6 but exactly 4 Time"
No counterexample found. Assertion may be valid.
\end{verbatim}
This result confirms that, under the modeled constraints, it is impossible for a trip to bypass the recording phase and appear directly in the saved history (Recorded state).

\paragraph{Simulation}
To visually verify the model behavior, we ran the \texttt{EventuallyRecorded} predicate. Figure \ref{fig:alloy_dynamic} demonstrates a valid execution trace generated by the analyzer, showing the state of a Trip object evolving over distinct time steps.

\begin{figure}[H]
    \centering
    % This corresponds to your uploaded image: image_69c57b.png
    % Please rename it to alloy_dynamic.png
    \includegraphics[width=1.0\textwidth]{alloy_dynamic.png}
    \caption{Alloy Trace: Dynamic evolution of the Trip State over Time}
    \label{fig:alloy_dynamic}
\end{figure}

% Alloy model and assertions

% =================================================
